\documentclass[a4paper,titlepage,DIV11,11pt,BCOR2.5cm,headinclude,ngerman,openany,pdftex]{article}

\usepackage[utf8]{inputenc} %input encoding: UTF8 (with every modern LaTeX Editor UTF8 encoding should work)
%\usepackage[latin1]{inputenc}  % under windows some editors produce source files with latin1 encoding

%language options:
\usepackage[ngerman]{babel} %use new german language for labels

%quotation
\usepackage{csquotes}

%tables
\usepackage{tabu}

%Images / floats:
\usepackage[final]{graphicx} %handling of graphics
\usepackage{rotating} %rotation of graphics / sideways figures
\usepackage[bf]{caption} %changes style of figure / floating object captions, bold figure number
\graphicspath{{images/}} %Bilder aus diesem Ordner laden

%change font styles
\usepackage{fourier} %fourier font
\usepackage{sectsty} %allows to change the font of section headings
\allsectionsfont{\fontfamily{phv}\selectfont} %select helvetica for the section headings

\usepackage[left=30mm,right=25mm,top=25mm,headsep=15mm]{geometry}
\usepackage[onehalfspacing]{setspace}


\usepackage{varioref} %nice references (optional references with page number)
\usepackage{color}
\usepackage{listings} %nice typesetting of source code listings
\lstset{language=Ruby, frame=single, breaklines=true, numbers=left, basicstyle=\footnotesize, morekeywords={var, yield}, showspaces=false, showtabs=false, tabsize=2}

%output configuration of the pdftex package / configuration of the generated pdf document:
 \usepackage[pdftex,
    colorlinks=true,
    linkcolor=blue,
    filecolor=blue,
    citecolor=cyan,
    pdftitle={Performance-Analyse und -Optimierung bei
    Single-Page-Anwendungen auf Basis von Angular.JS},
    pdfauthor={Thomas Fiedler},
    pdfsubject={performance, javascript, angularjs},
    pdfkeywords={javascript, performance, angularjs},
    bookmarks, bookmarksnumbered=true]{hyperref}

\usepackage{hyphenat}
\hyphenation{}

\definecolor{fzp}{rgb}{0.565,0.612,0.671}
\usepackage{fancyhdr}
\setlength{\headheight}{37.9pt}
\pagestyle{fancy}
\fancyhf{}
%Kopfzeile links bzw. innen
\fancyhead[L]{\small{Universtiät Leipzig}}
%Kopfzeile mittig
\fancyhead[C]{}
%Kopfzeile rechts bzw. außen
%\fancyhead[R]{\includegraphics{images/jira-icon.png}}
%Linie oben
%\renewcommand{\headrulewidth}{2pt}
\renewcommand\headrule
{{\color{fzp}%
	\hrule height 3pt width \headwidth
	\vspace{1pt}%
	\hrule height 1pt width \headwidth
	\vspace{-4pt}
}}
%Fußzeile rechts
\fancyfoot[R]{\thepage}
%Fußzeile Plain
\fancypagestyle{plain}{ %
  \fancyhf{} % remove everything
  \renewcommand{\headrulewidth}{0pt} % remove lines as well
  \renewcommand{\footrulewidth}{0pt}
  \renewcommand{\headrule}{}
  \fancyfoot[R]{\thepage}
}

\setcounter{secnumdepth}{3} %set deepness of numbered sections

%Biblatex
\usepackage[
backend=biber,
style=chicago-authordate,
sorting=ynt
]{biblatex}

\addbibresource{./literatur.bib} %Imports bibliography file

%new commands / macros:

%------------------------------------------ DOC START ------------------------
\begin{document}

% ---TITELSEITE---
\begin{titlepage}

\begin{center}
{\Large \bfseries Berufsakademie Leipzig}

\vspace{1,0cm}
\sffamily
{\LARGE \bfseries Bachelor-Arbeit}\\
\vspace{1,5cm}
{\Huge  \fontfamily{phv}\selectfont  Performance-Analyse und -Optimierung bei
Single-Page-Anwendungen auf Basis von Angular.JS}\\
\vspace{2,0cm}
\vspace{4,0cm}
\small
ausgeführt am IALT der\\
Universität Leipzig\\
\vspace{1,0cm}
durch\\
\vspace{,2cm}
Thomas Fiedler \\
Mat.Nr.: 5000345\\
\vspace{,5cm}

\end{center}

\end{titlepage}


%manuell leere Seite erzwingen
%\mbox{} \thispagestyle{empty} \newpage

%--- TITELSEITE ENDE ---

\pagenumbering{roman}

% ---ZUSAMMENFASSUNG ---
%\section*{Zusammenfassung}
%Hier kommt die Zusammenfassung hin (wenn es eine gibt)

% --- TABLE OF CONTENTS ---
\tableofcontents
\newpage
% -- list of figures --
%\listoffigures
%\newpage
\pagenumbering{arabic}
%\pagestyle{headings}

% ----- BEGINN HAUPTTEIL -----
\section{Einleitung}
% Ziel+Anliegen, Methoden

Beantwortung der Frage \emph{Why performance matters} und so weiter...
Bla bla bla bla bla bla bla

\section{Metriken}

- Möglichkeiten zur kontinuierlichen Erfassung/Kontrolle geeigneter Metriken
- Betrachtung von existierenden Werkzeugen
  - ,,Chrome Dev Tools``
    - Timeline
    - Flamecharts
  - WebPageTester
  - Batarang

% ----- ANHANG -----
\include{anhang}

%% -- abkürzungen --
%\include{masterThesisAbbrev}
%\printglossary
%\addcontentsline{toc}{chapter}{Abkürzungsverzeichnis}

%%  -- index --
%\renewcommand{\indexname}{Sachregister}
%\printindex
%\addcontentsline{toc}{chapter}{Sachregister}

% -- list of tables --
%\listoftables


%% -- literaturverzeichnis --
\printbibliography[
heading=bibintoc,
title={Bibliographie}
]
%\addcontentsline{toc}{section}{Literaturverzeichnis}

% ---Erklärung---
\section*{Erklärung}
Ich versichere, dass ich die vorliegende Arbeit ohne fremde Hilfe
selbstständig verfasst und nur die angegebenen Quellen und
Hilfsmittel benutzt habe. Wörtlich oder dem Sinn nach aus anderen
Werken entnommene Stellen sind unter Angabe der Quellen kenntlich
gemacht. Die Arbeit wurde bisher in gleicher oder ähnlicher Form
weder veröffentlicht, noch einer anderen Prüfungsbehörde vorgelegt.
 \\[14cm]

%TODO: hardcode the finishing date here
\today \hspace{5.3cm} Rebecca Tomczyk

\end{document}
%----------------------------------------------------------------------------
